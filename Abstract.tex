%Abstract Page 

\hbox{\ }

\renewcommand{\baselinestretch}{1}
\small \normalsize

\begin{center}
\large{{ABSTRACT}} 

\vspace{3em} 

\end{center}
\hspace{-.15in}
\begin{tabular}{ll}
Title of dissertation:    & {\large  PREVENTING BUFFER OVERFLOWS}\\
&				      {\large  WITH BINARY REWRITING} \\
\ \\
&                          {\large  P\'{a}draig O'Sullivan, Master of Science, 2010} \\
\ \\
Thesis directed by: & {\large  Professor Rajeev Barua} \\
&  				{\large	 Department of Electrical and Computer Engineering} \\
&  				{\large	 Professor Angelos Keromytis} \\
&  				{\large	 Department of Computer Science, Columbia University} \\
\end{tabular}

\vspace{3em}

\renewcommand{\baselinestretch}{2}
\large \normalsize

Buffer overflows are the single largest cause of security attacks in recent times. Attacks based on
this vulnerability have been the subject of extensive research and a significant number of defenses
have been proposed for dealing with attacks of this nature. However, despite this extensive
research, buffer overflows continue to be exploited due to the fact that many defenses proposed in
prior research provide only partial coverage and attackers have adopted to exploit problems that are
not well defended. The fact that many legacy binaries are still deployed in production environments
also contributes to the success of buffer overflow attacks since most, if not all, buffer overflow
defenses are source level defenses which require an application to be re-compiled. For many legacy
applications, this may not be possible since the source code may no longer be available. In this
thesis, we present an implementation of a defense mechanism for defending against various attack
forms due to buffer overflows using binary rewriting. We study various attacks that happen in the
real world and present techniques that can be employed within a binary rewriter to protect a binary
from these attacks.

Binary rewriting is a nascent field and little research has been done regarding the applications of
binary rewriting. In particular, there is great potential for applications of binary rewriting in
software security. With a binary rewriter, a vulnerable application can be immediately secured
without the need for access to it's source code which allows legacy binaries to be secured. Also,
numerous attacks on application software assume that application binaries are laid out in certain
ways or have certain characteristics. Our defense scheme implemented using binary rewriting
technology can prevent many of these attacks. We demonstrate the effectiveness of our scheme in
preventing many different attack forms based on buffer overflows on both synthetic benchmarks and
real-world attacks.
