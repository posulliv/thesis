%Chapter 4

\renewcommand{\thechapter}{4}

\chapter{Experimental Results}

\section{Synthetic Results}

In order to test how effective our scheme is, we utilized the benchmarks provided by Wilander and
Kamkar \cite{}.

\subsection{Benchmark Description}

Wilander and Kamkar developed twenty buffer overflow attack forms in order to evaluate the
effectiveness of tools available at the time that aimed to stop buffer overflows. An attack form is
defined as a combination of a technique, location, and an attack target. These terms are in turn
defined by Wilander and Kamkar as:

\begin{itemize}

\item \textbf{Technique} - either the buffer is overflowed all the way to the attack target
or the buffer is overflowed to redirect a pointer to the target

\item \textbf{Location} - the types of location for the buffer overflow are the stack or the
heap/BSS/data segment

\item \textbf{Attack target} - there are four targets - 1) the return address, 2) the old
base pointer, 3) function pointers, and 4) longjmp buffers

\end{itemize}

Considering all practically possible combinations gives the twenty attack forms listed below:

\begin{enumerate}

\item Buffer overflow on the stack all the way to the target

  \begin{enumerate}
  \item Return address
  \item Old base pointer
  \item Function pointer as a local variable
  \item Function pointer as parameter
  \item Longjmp buffer as local variable
  \item Longjmp buffer as function parameter
  \end{enumerate}

\item Buffer overflow on the heap/BSS/data segment all the way to the target

  \begin{enumerate}
  \item Function pointer
  \item Longjmp buffer
  \end{enumerate}

\item Buffer overflow of a pointer on the stack and then pointing at target

  \begin{enumerate}
  \item Return address
  \item Old base pointer
  \item Function pointer as a local variable
  \item Function pointer as parameter
  \item Longjmp buffer as local variable
  \item Longjmp buffer as function parameter
  \end{enumerate}

\item Buffer overflow of a pointer on the heap/BSS/data segment and then pointing at target

  \begin{enumerate}
  \item Return address
  \item Old base pointer
  \item Function pointer as a local variable
  \item Function pointer as parameter
  \item Longjmp buffer as local variable
  \item Longjmp buffer as function parameter
  \end{enumerate}

\end{enumerate}

Of the twenty attack forms, we obtained the source code to only eighteen of these attack targets.

\subsection{Methodology}

We compiled the benchmarks using gcc 4.4. We compiled two versions of the benchmarks - one version
had the \begin{em}-fno-stack-protector\end{em} flag while the other had the
\begin{em}-fstack-protector\end{em} flag. The \begin{em}-fstack-protector\end{em} flag creates a
binary with the ProPolice protection mechanism embedded within it.

\subsection{Results and Analysis}

\section{Overheads}

\subsection{Binary Rewriting Overhead}

\subsection{Overhead of the Scheme}

\section{Real World Attacks}

Ultimately, the success of the scheme presented in this thesis is only valuable if it is applicable
to real-world attacks i.e. whether it can prevent attacks that have been observed in practice. In
this section, we first analyze US-CERT advisories that have been issued in the last five years.
Next, we reproduce a real-world attack and demonstrate that our rewriter halts this attack.

\subsection{US-CERT Advisories}

The Computer Emergency Response Team, or CERT, was formed at CMU in the late 80s and begin releasing
"Advisories" that described vulnerabilities in order to alert system administrators of potential
security problems. These advisories have now become a core part of US-CERT's technical cyber
security alerts \cite{}.

\subsection{Stack Buffer Overflow Attack Against HTTP Server}

A HTTP srever, GHTTPD, has a stack buffer overflow vulnerability in its logging function. We
produced an exploit for this server which overflows a stack-based buffer and corrupts the return
address.
