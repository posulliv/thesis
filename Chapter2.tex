%Chapter 2

\renewcommand{\thechapter}{2}

\chapter{Related Work}

Our work is related to many techniques that attempt to defend against malicious applications and
vulnerabilities in applications which present attackers with an opportunity to attack. In this
chapter, we elaborate on some of the pieces of work most closely related to ours. First, we present
the various attack techniques utilized by attackers that are relevant for this thesis and then we go
on to present various techniques proposed for mitigating these attack techniques. When discussing
defenses, we draw attention to the drawbacks of each technique. We also briefly discuss related work
in binary rewriting.

\section{Catalog of Attack Techniques}

\subsection{Buffer Overflow Attacks}

A buffer overflow refers to a situation that can occur when a function contains a local bounded
array, or buffer, and writes into that buffer are not correctly guarded. Since C and C++ compilers
typically use the stack for local variables as well as parameters, frame pointers, and saved return
addresses, writes into a buffer that are not correctly guarded may overwrite and corrupt the return
address stored on the stack. Data copied into the buffer whose length is larger than the buffer's
size is referred to as a buffer overflow.

If an attacker controls the data used by the function, attackers can exploit buffer overflows and
change the function's return address to an arbitrary value. In this case, when the function returns,
the attacker can direct execution to code of their choice. This technique was first described in
detail by AlephOne in 1996 \cite{}. However, attacks of this kind date back to before 1988 when the
technique was used in the \begin{em}fingerd\end{em} exploit of the Morris worm.

An example of a function with a buffer overflow vulnerability adopted from \cite{erlingsson-07} follows:

\begin{mylisting}
\begin{verbatim}
int is_file_my_file(char *one, char *two)
{
  char tmp[MAX_FILE_LEN];
  strcpy(tmp, one);
  strcat(tmp, two);
  return (strcmp(tmp, "/home/posulliv/my_file"));
}
\end{verbatim}
\end{mylisting}

The above (somewhat unrealistic) function compares the concatentation of two input strings against
\begin{em}/home/posulliv/my\_file\end{em}. If the input strings to the above function can be chosen by an
attacker, then the attacker can change the program's flow of control by overflowing the
\begin{em}tmp\end{em} buffer and changing the return address stored on the stack to an address of
the attacker's choosing.

Commonly, an attacker would choose their input data so that the machine code for an attack payload
would be present at the modified return address. When the vulnerable function returns, and execution
of the attack payload begins, the attacker has gained control of the behavior of the target
software. The attack payload is often called shellcode, since a common goal of an attacker is to
launch a command line interpreter (referred to as a shell in UNIX like environments) under their
control.

\subsection{Return-to-libc Attacks}

As an alternative to supplying executable code (referred to as direct code injection), an attacker
might be able to craft an attack that executes existing machine code (indirect code injection). This
class of attacks has been referred to as jump-to-libc or return-to-libc (arc injection \cite{} has
also been used to refer to this class of attacks) because the attack often involves directing
execution towards machine code in the standard C library (libc) \cite{}. The standard C library is
often the target for attacks of this type since it is loaded in nearly every UNIX program and it
contains routines of the sort that are useful for an attacker. This technique was first suggested by
Solar Designer in 1997 \cite{}. Attacks of this kind can evade defense mechanisms that protect the
stack such as stack canaries and it is also effective against defenses that only allow memory to be
writable or executable.

An example of a function vulnerable to such an attack adopted from \cite{erlingsson-07} follows:

\begin{mylisting}
\begin{verbatim}
int median(int *data, int len, void *cmp)
{
  int tmp[MAX_INTS];
  /* copy the input integers */
  memcpy(tmp, data, len * sizeof(int));
  /* sort the local copy */
  qsort(tmp, len, sizeof(int), cmp);
  /* median is in the middle */
  return tmp[len / 2];
}
\end{verbatim}
\end{mylisting}

The above function is vulnerable to a buffer overflow as outlined in the previous sub-section.
However, an attacker can also corrupt the comparison function pointer cmp before it is passed to the
\begin{em}qsort\end{em} library function. If an attacker is able to accomplish this, he/she can gain control of
execution at the point where the \begin{em}qsort\end{em} function calls its copy of the corrupted
\begin{em}cmp\end{em} argument.

Traditionally, attacks of this kind have targeted the system function in the standard system library
which allows the execution of an arbitrary command with arguments. However, recent attacks have been
demonstrated which do not depend on calling functions in the standard C library. 

\subsection{Return-Orientated Programming}

The technique of return-oreintated programming was introduced by Shacham et al \cite{}. Using this
technique, an attacker can induce arbitrary behavior in a program whose control flow he/she has
diverted without injecting any code. A return-orientated program chains together short instruction
sequences already present in a program's address space, each of which ends in a return instruction.
Several instructions can be combined into a gadget which is the basic block within a
return-orientated program that performs operations. Gadgets are self-contained and perform one
well-defined step of a computation. An attacker uses these gadgets to craft stack frames that can
then perform arbitrary computations. 

Shacham et al. showed that the standard C library in both Linux running on the x86 platform and
Solaris running on the SPARC platform contain enough useful intructions to construct meaningful
gadgets. They manually analyzed the standard C library on both platforms and constructed a library
of gadgets that is Turing complete.

\section{Catalog of Defense Techniques}

\subsection{Compile Time Defenses}

StackGuard \cite{stackguard-98} places a 'canary' on the stack between local variables and the
return address. This canary value is designed to warn of stack corruption since validating the
integrity of the canary value is an effective means of ensuring that the function return address has
not been corrupted. Microsoft's compiler also supports the insertion of stack canaries with the /GS
option.

ProPolice \cite{} is similar to StackGuard in that it places a canary value on the stack. However,
ProPolice also places arrays and other function-local buffers above all other function-local
variables on the stack. Copies of all function arguments are also made into new, function-local
variables that also sit below any buffers in the function. As a result, these variables and
arguments are not subject to corruption through an overflow of these buffers. 

StackGuard, PointGuard, and ProPolice involve compile-time analysis and transformation. Thus, unless
the source code for an application is available, these techniques can not be used thereby hindering
the ability to easily deploy these techniques.

\subsection{Instruction Set Randomization}

Instruction-set randomization \cite{} is a promising technique for protecting against buffer
overflows (and many kinds of code injection attacks). This approach randomizes the underlying
system's instructions so that foreign code injected by an attacker would fail to execute correctly
since the attacker does not know the instruction set of the target system. However, as mentioned by
the authors in \cite{}, the main drawback of this technique as applied to binary code that is meant
to execute on a hardware processor is the need for special support by the processor. Thus, even
though instruction-set randomization offers a strong defense against buffer overflow attacks the
fact that unless it is supported by specialized hardware, it incurs significant overheads means that
it is inlikely to see adoption in practice for the foreseeable future. However, when applied to
interpreted languages (such as SQL), it can prove to be very effective. However, interpreted
languages are not vulnerable to the buffer overflow attacks being discussed in this thesis.

\section{Related Work in Binary Rewriting}

Binary rewriting and link time optimizatizers have been considered by a number of researchers.
Binary rewriting research is being carried out in two directions: static rewriting and dynamic
rewriting. Dynamic binary rewriters rewrite the binary during its execution. Examples are PIN
\cite{}, BIRD \cite{}, DynInst \cite{}, DynamoRIO \cite{} and Valgrind \cite{}. None of the dynamic
binary rewriters we found employ a compiler like intermediate representation. This is not surprising
since dynamic rewriters construct their internal representation at run-time, and hence they would
not have the time to construct a compiler IR. Dynamic rewriters are hobbled since they do not have
enough time to perform complex compiler transformations either; they have been primarily used for
code instrumentation and simple security checks in the past. We do not discuss dynamic rewriters
further since the methods used for rewriting in this thesis are primarily directed at static binary
rewriters such as SecondWrite.

Existing static binary rewriters related to our approach include Etch \cite{}, ATOM \cite{}, PLTO
\cite{}, and Diablo \cite{}. None of the existing rewriters employ a compiler level intermediate
representation; rather they define their own low-level custom intermediate representation. Diablo
defines an augmented whole program control-flow-graph based intermediate representation with program
registers as globals and memory as a black box. It does not attempt to obtain high-level information
like function prototypes and is geared mainly towards optimizations like code compaction. Taking
memory as a black box limits its applicability to architectures like x86 which have a limited set of
registers. ATOM defines a symbolic RTL-based intermediate representation with infinite registers but
does not make any attempt of analyzing or modifying the stack layout. Its mainly targeted towards
RISC architectures. PLTO employs a whole program CFG based IR and implements stack analysis to
determine the use-kill depths of each function \cite{}. However, this information is not used for
converting it into high-level IR; rather it is only used for low-level custom optimizations like
load/store forwarding. Etch does not explicitly build an intermediate representation build an
intermediate representation and allows users to add new tools to analyze binaries. The primary goal
of Etch appears to be instrumentation and has only been shown to be applicable for simple
optimizations like profile-guided code layout.

