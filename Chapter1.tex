%Chapter 1

\renewcommand{\thechapter}{1}

\chapter{Introduction}

The dominant form of software security vulnerability is the buffer overflow vulnerability. Attacks
based on this vulnerability have been the subject of extensive research and a significant number of
defenses have been proposed for dealing with attacks of this nature. Despite this extensive
research, buffer overflows continue to be exploited in real world attacks. This is because most
buffer overflow defenses provide only partial coverage, and the attacks have adopted to exploit
problems that are not well defended.

Numerous schemes have been proposed for dealing with buffer overflow attacks. In our eyes, the
applicability of such schemes depends on a number of factors, including:

\begin{enumerate}
 \item Ease of use
 \item Availability
 \item Robustness
 \item Applicability to all binaries
 \item Low overheads
\end{enumerate}

In this thesis, we present a scheme which meets the applicability requirements for a good security
scheme. Our scheme is implemented using a binary rewriter developed within our research group named
SecondWrite. A binary rewriter takes a binary executable program as input, and produces a modified,
improved executable as output. 

Two of the novel aspects of SecondWrite are: 1) the input binary is translated into an existing
compiler's intermediate representation (IR); and 2) binaries without relocation or symbolic
information can be rewritten. 

The ability to translate a binary to a high-level IR allows SecondWrite to insert security
mechanisms that would otherwise require access to source code. Using a rewriter to insert security in a binary is important for
consumers/users of software, who are otherwise at the mercy of the vendor when it comes to using
security mechanisms or fixing known problems (patch management). With a binary rewriter, an
administrator can (in principle) modify a binary to fix or mitigate a vulnerability. 

\section{Contributions}

The contributions of this thesis are primarily in demonstrating a key application of binary
rewriting. To this end, this thesis makes two major contributions:

\begin{enumerate}

 \item A scheme is presented for protecting against commong buffer overflow attacks using a
 sophisticated and novel static binary rewriter

 \item The scheme presented is practical, effective, and immediately deployable

\end{enumerate}

\section{Outline}

In Chapter 2, we discuss some of the common attack forms and survey existing defenses for these
attacks. We also briefly discuss some previous work in the field of binary rewriting. In Chapter 3,
we describe how our binary rewriter works and how the various components of our scheme are
implemented. In Chapter 4, we present experimental results which demonstrate the practicality of our
scheme. Finally, in Chapter 5, we present our conclusions and discuss directions for future work.
