%Chapter 3

\renewcommand{\thechapter}{3}

\chapter{Overview of Scheme}

The cornerstone of our scheme is a binary rewriter which has been developed within our research
group. In this chapter, we first discuss binary rewriting, give an overview of the binary rewriter
we have developed in our research group, and then go on to discuss the various components of our
scheme that we have implemented as part of our binary rewriter.

\section{Binary Rewriting}

Binary rewriters are pieces of software that accept a binary executable program as input, and
produce an improved executable as output. The output executable typically has the same functionality
as the input, but is improved in one or more metrics, such as run-time, energy use, memory use,
security or reliability. 

In recognition of its potential, binary rewriting has seen much active research over the last
decade. The reason for great interest in this area is that binary rewriting offers additional
advantages over compiler-produced optimized binaries:

\begin{itemize}

\item Ability to do inter-procedural optimization.

\item Ability to do optimizations missed by the compiler.

\item Increased economic feasibility.

\item Portable to any source language and any compiler.

\item Works for hand-coded assembly routines.

\end{itemize}

However, binary rewriters today have fallen far short of this desired vision. Binary rewriters
remain relatively crude tools today, capable of no more than simple program transformations such as
peephole optimization and code instrumentation. Complex transformations such as extensive
whole-program optimizations, automatic parallelization and sophisticated security enforcement, which
we study in this thesis, remain outside the capabilities of current rewriters.

The binary rewriter developed by our group and utilized for this research is named SecondWrite. Our
binary rewriter employs the widely used open-source LLVM compiler infrastructure and in particular,
LLVM's high-level intermediate representation to represent code. Our custom binary reader and
de-compiler modules read a binary and produce requivalent LLVM IR code using some of the techniques
we will briefly describe in Section 3.1.1.

For this thesis, we study using binary rewriting to retroactively add security to a vulnerable
binary. When this extra security is added, a binary is no longer vulnerable to common buffer
overflow attacks.

Two notable properties of using binary rewriting to enforce security are low-overhead and real-time
prevention of malicious behaviors as will be seen when we present our experimental results.

\subsection{Architecture of Binary Rewriter}

Figure \ref{} presents an overview of the SecondWrite system. The SecondWrite system consists of a
frontend module for reading binary executables and generating an initial LLVM IR, an internal pass
module for extracting more information about the underlying program, optimizing passes to implement
various optimizations, and the LLVM code generator (codegen) for producing the rewritten binary.

\begin{figure}
\begin{center}
\includegraphics[0in,0in][3.25in,4.266in]{sw_overview.eps}
\caption{SecondWrite system}
\end{center}
\end{figure}

The frontend module consists of a disassembler and a custom binary reader which processes the
individual instructions and generates an initial LLVM IR. This initial representation is void of the
desired IR features like function prototypes, abstract stack and virtual registers. The internal
pass module analyzes this initial IR to obtain an improved IR which has all the information and
features mentioned previously. Various optimization passes can be written on the above IR to obtain
an optimized IR. Finally, the optimized IR is passed to the existing LLVM code generator to obtain
the rewritten binary.

Various inherent characteristics of executables such as the unavailability of function prototypes,
the use of a phyiscal stack and the use of the set of phyiscal registers make it difficult to obtain
a high-level IR from an input executable. A number of techniques have been developed within our
group to extract this high-level information from executables whenever possible. We will not discuss
those techniques in this thesis as the techniques are explained in detail elsewhere \cite{}.


\section{Return Address Protection}

\section{Function Pointer Protection}

\section{longjmp/setjmp Protection}
