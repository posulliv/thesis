%Chapter 3

\renewcommand{\thechapter}{3}

\chapter{Overview of Scheme}

The cornerstone of our scheme is a binary rewriter which has been developed within our research
group. In this chapter, we first discuss binary rewriting and then go on to discuss the various
components of our scheme which are implemented as part of our binary rewriter.

\section{Binary Rewriting}

Binary rewriters are pieces of software that accepts a binary executable program as input, and
produce an improved executable as output. The output executable usually has the same functionality
as the input, but is improved in one or more metrics; in our case, security. In our research, we use
binary rewriting to retroactively add security to a vulnerable binary. When this extra security is
added, a binary is no longer vulnerable to common buffer overflow attacks.

Two notable properties of using binary rewriting to enforce security are low-overhead and real-time
prevention of malicious behaviors. 

\subsection{Architecture of Binary Rewriter}

\section{Return Address Protection}

\section{Function Pointer Protection}

\section{longjmp/setjmp Protection}
